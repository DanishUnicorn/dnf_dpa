\chapter{Literature résumés}
\setlength{\headheight}{12.71342pt}
\addtolength{\topmargin}{-0.71342pt}

This section of the course notes is designed to streamline access to the key findings from each reading material (RM), providing a concise and accessible overview of essential information. Created through experimentation with various AI platforms, this chapter also serves to enhance prompt engineering skills, exploring diverse methods of note-taking for maximum efficiency and clarity. The procedures for creating these summaries have varied, but all methods share a common approach: each RM has been fully read, with summaries and notes prepared after completing each respective subsection. By using these AI-co-op'ed approaches, these notes aim to be both a reliable reference and a resource for continuous improvement in capturing complex microbiology concepts.

\section{1\texorpdfstring{\textsuperscript{st}}{st} Reading Material from the Curriculum}

\subsection{Milk for Liquid Consumption}
Liquid milk is treated by \textbf{pasteurization} or \textbf{sterilization} to ensure \textbf{safety}, extend \textbf{shelf life}, and retain \textbf{flavour}. Raw milk is considered unsafe and is restricted in many countries. \textbf{Pasteurized milk} retains better flavour, while \textbf{sterilized milk} offers longer shelf life, especially valued in cooking. \textbf{Fat content} is usually standardized, but \textbf{low-fat} and \textbf{skim milks} are also common. Some products are \textbf{fortified} or processed via \textbf{ultrafiltration} for consistent protein content, although this may be legally restricted. Quality attributes vary by use, and packaging is essential for hygiene \cite*{curr_rm_01_dairy_science_technology}.

\subsubsection*{Manufacture}
\textbf{Thermalization} reduces lipase activity and psychrotrophic growth, aiding shelf life. \textbf{Homogenization} prevents creaming but increases \textbf{lipolysis risk}, requiring higher heat (e.g., 20 s at 75~\textdegree C). \textbf{Low pasteurization} (15 s at 72~\textdegree C) kills pathogens while preserving \textbf{natural inhibitors}, though heat-sensitive compounds like \textbf{agglutinins} and \textbf{immunoglobulins} are degraded in \textbf{high-pasteurized milk}. \textbf{Packaging hygiene} is crucial to prevent recontamination and preserve shelf life \cite*{curr_rm_01_dairy_science_technology}.


\subsubsection*{Shel Life}
\textbf{Shelf life} is influenced by \textbf{bacterial growth}, \textbf{enzymatic activity}, and \textbf{chemical/physical changes}. Key factors include \textbf{storage temperature}, \textbf{recontamination}, and \textbf{Bacillus cereus} spore levels. Below 7~\textdegree C, psychrotrophs dominate spoilage. Hygiene in \textbf{packaging} is essential; \textbf{rapid tests} help detect recontamination \cite*{curr_rm_01_dairy_science_technology}.

\subsubsection*{Extended-Shelf-Life Milk}
\textbf{ESL milk} combines long shelf life with near-fresh flavour. One method applies \textbf{short-time direct UHT treatment} (e.g., 2 s at 140~\textdegree C) with \textbf{aseptic packaging}; enzymes like \textbf{plasmin} may still affect taste after weeks. The second method involves \textbf{microbial removal} via \textbf{microfiltration} or \textbf{bactofugation}, often followed by partial \textbf{UHT sterilization} of retentate and cream. Aseptic packaging is essential. \textbf{Cooked flavour} is minimized by limiting heat to fat-rich fractions \cite*{curr_rm_01_dairy_science_technology}. 

\subsection{Sterilized Milk}
\subsubsection{Description}
\textbf{Sterilized milk} must be microbe-free, shelf-stable at ambient temperature, and retain \textbf{acceptable flavour} and \textbf{nutritional value}. \textbf{UHT sterilization} (e.g., 1 s at 145~\textdegree C) minimizes \textbf{browning}, \textbf{off-flavours}, and \textbf{vitamin loss}. To prevent spoilage, packaging must be \textbf{aseptic}, and milk free from heat-resistant enzymes. \textbf{Homogenization} avoids creaming and coalescence. \textbf{Lactulose} content is used to identify UHT-treated milk \cite*{curr_rm_01_dairy_science_technology}.

\subsubsection*{Manufacture}
\textbf{Sterilized milk} is made via \textbf{in-bottle}, \textbf{mild in-bottle}, or \textbf{flow-through UHT} processes. \textbf{Psychrotroph enzymes} (esp. from \textit{Pseudomonas}) are heat-resistant, so raw milk must be fresh. \textbf{UHT heating} (>140~\textdegree C) ensures safety but risks \textbf{casein aggregation}, off-flavors, and \textbf{vitamin loss}. \textbf{Aseptic homogenization} and \textbf{deaeration} are crucial to prevent oxidized flavor. \textbf{Oxygen- and light-tight packaging} prolongs shelf life \cite*{curr_rm_01_dairy_science_technology}. 

\subsubsection*{Shelf Life}
Spoilage of \textbf{in-bottle sterilized milk} may result from surviving \textbf{spores} (e.g., \textit{B. subtilis}, \textit{B. stearothermophilus}) or \textbf{leaky packaging}. \textbf{UHT milk} mainly deteriorates via \textbf{recontamination} or residual \textbf{heat-resistant enzymes}, causing \textbf{gelation}, \textbf{off-flavors}, or \textbf{plasmin-induced bitterness}. \textbf{Nonenzymatic spoilage} includes \textbf{oxidation}, \textbf{Maillard reactions}, and \textbf{light effects}. Shelf life is tested via \textbf{incubation}, \textbf{oxygen pressure}, or \textbf{ATP bioluminescence} \cite*{curr_rm_01_dairy_science_technology}.

\subsection{Reconstituted Milk}
\textbf{Reconstituted milk} is made by dissolving milk powder in water; \textbf{recombined milk} adds \textbf{anhydrous milk fat} to reconstituted skim milk. It mimics whole milk but lacks \textbf{natural fat globule membrane} components. \textbf{Filled milk} uses \textbf{vegetable oil} instead of milk fat. \textbf{Toned milk} blends \textbf{buffalo milk} with skim milk to reduce fat content \cite*{curr_rm_01_dairy_science_technology}. 