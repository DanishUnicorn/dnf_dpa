\chapter*{Course Description}
\setlength{\headheight}{12.71342pt}
\addtolength{\topmargin}{-0.71342pt}

\section*{Content}
The course deals with the product technologies of a range of non-fermented dairy products, e.g. fluid milk products, concentrated and dried dairy products, butter and spreads, as well as ice cream. Plant-based dairy analogues will also be introduced.

Furthermore, the course deals with unit operations and equipment applied in the production of dairy products as well as technological innovations, developments, optimizations and trends within well-established technologies, e.g. for improvement of environmental impact. 

\section*{Learning Outcome}
The overall aim of the course is to provide students with in-depth knowledge of the technologies used for processing of non-fermented dairy products.

\vline

After completing the course the student should be able to:

\textbf{Knowledge}
\begin{itemize}
    \item Discuss the major factors influencing the quality of dairy products and relevant manufacturing processes.
    \item Identify and critically examine the principle(s) underpinning unit operations needed for the production of selected dairy products.
\end{itemize}
\vline

\textbf{Skills:}
\begin{itemize}
    \item Apply principles of colloid science to processing of dairy products.
    \item Apply obtained knowledge of dairy technology to pilot plant practicals and analyses on dairy products.
    \item Analyse data obtained for characterization of processing effects on physical and chemical properties of dairy products.
    \item Reflect and critically examine the effect of various unit operations on dairy product quality.
    \item Reflect and critically examine how colloidal interactions can be controlled in order to produce stable dairy products.
\end{itemize}
\vline

\textbf{Competences:}
\begin{itemize}
    \item Discuss how the technology behind manufacture of dairy products affects final product quality.
    \item Discuss the contribution of different unit operations to product properties, and the value of different product streams in production of selected dairy products.
    \item Work effectively in a group for both theoretical and practical assignments.
\end{itemize}

\section*{Litterature}
See Absalon for a list of course literature. The curriculum will include lecture notes and scientific papers in addition to parts of textbooks.

\section*{Recommended Academic Qualifications}
Qualifications corresponding to having completed the Dairy Internship are recommended.

Academic qualifications equivalent to a BSc degree is recommended .


\section*{Teaching and Learning Methods}
Lectures, project work in the dairy pilot plant and laboratory and half/one-day excursions.

Minor costs for one-day excursions (e.g. transportation) are paid by the students.

\section*{Workload}

\begin{table}[h]
    \centering
    \caption{A table with an overview over the workload for the course.}
    \label{tab:workload}
    \rowcolors{2}{white}{gray!7}
    \begin{tabular}{ l | c}
        \textbf{Category} & \textbf{Hours} \\ 
        \hline
        Lectures & 30 \\ 

        Preparation & 88 \\

        Practical exercises & 12 \\ 

        Excursions & 8 \\

        Project work & 55 \\

        Guidance & 12 \\

        Exam & 1 \\ 
        \hline
        Total & 206 \\ 
    \end{tabular}
\end{table}





\newpage