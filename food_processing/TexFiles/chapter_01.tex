\chapter{Course Description}
\section{Content}
The course deals with understanding the main unit operations and equipment involved in the manufacture of foods and ingredients (i.e. mixing, heating, baking, drying, extrusion, separation, etc.), as well as aspects of sustainability in food processing, applications of sensor technology and process control.

Emphasis is on understanding the individual food processing steps with respect to obtaining optimal process conditions and desired product characteristics, and on the integration of the different unit operations in a production line. The course also aims to demonstrate how HACCP can be employed in food processing to obtain safe foods.

\section{Learning Outcome}
The overall aim of the course is to provide the students with knowledge on unit operations, equipment and production processes applied in the food industry for manufacturing of foods and ingredients.

\subsection{Knowledge}
\begin{highlight}
    \begin{itemize}
        \item Describe the principles of unit operations commonly used in the manufacturing of foods
        \item Identify how to improve product properties and sustainability in food processing
        \item Identify process steps suitable for process analysis and control
        \item Describe food safety management, using HACCP.
    \end{itemize}
\end{highlight}

\subsection{Skills}
\begin{highlight}
    \begin{itemize}
        \item Apply theory to solve theoretical problems regarding process equipment, process design and optimization
        \item Select measures to ensure correct process functionality (product properties, energy or water savings)
        \item Analyse unit operations in the food industry and apply knowledge of physics and chemistry to them
        \item Use theory to solve and explain practical problems as well as analyze process data
        \item Apply HACCP for food safety management.
\end{itemize}
\end{highlight} 

\subsection{Competences}  
\begin{highlight}
    \begin{itemize}
        \item Reflect on the interplay between unit operations along a process line
        \item Evaluate whether existing and/or new control strategies are appropriate in order to achieve safe and robust food products.
        \item CDiscuss the application of different types of equipment and process steps to obtain food products with specific, desired properties
        \item Be able to describe process lines for a number of foods with required equipment.
    \end{itemize}
\end{highlight}

\section{Litterature}
See Absalon for a list of course literature.

\section{Recommended Academic Qualifications}
It is expected that the student has a basic understanding of food-related physics, microbiology and food chemistry. In the first weeks, the students will be divided in two groups depending on their previous knowledge in heat, mass transport and understanding of basic food unit operations.

Academic qualifications equivalent to a BSc degree is recommended.

\section{Teaching and Learning Methods}
The course will be based on lectures, theoretical exercises, short demonstrations and excursions may be organised. Some topics will be covered by external lecturers from the food industry. In case of excursions: minor costs for one-day excursions (e.g. transportation) are paid by the students.

\section{Remarks}
Expenses in relation to excursions will be subject to charge.

\section{Workload}
\begin{table}
    \centering
    \caption{A table with an overview over the workload for the course.}
    \label{tab:workload}
    \rowcolors{2}{white}{gray!7}
    \begin{tabular}{ l | c}
        \textbf{Category} & \textbf{Hours} \\ 
        \hline
        Lectures & 45 \\ 

        Preparation & 100 \\

        Theory exercises & 25 \\ 

        Excursions & 7 \\

        Project work & 25 \\

        Exam & 4 \\ 
        \hline
        Total & 206 \\ 
    \end{tabular}
\end{table}

\section{Feedback Form}
\begin{highlight}
    \begin{itemize}
        \item Oral
        \item Individual
        \item Collective
        \item Continuous feedback during the course of the semester
    \end{itemize}
\end{highlight}

\section{Sign Up}
Self Service at KUnet

http://www.science.ku.dk/english/courses-and-programmes/

https://www.science.ku.dk/english/continuing-and-professional-education/single-subject-courses/practical/

\section{Exam}
\newpage
\begin{table}[t]
    \centering
    \caption{The table shows the details of the course exam, as defined from the website of the University of Copenhagen.}
    \label{tab:course_details}
    \rowcolors{2}{white}{gray!7}
    \begin{tabular}{ l | >{\raggedright\arraybackslash}p{\textwidth - 5.8cm} }
        \textbf{Category} & \textbf{Details} \\ 
        \hline
        Credit & 7.5 ECTS \\ 

        Type of assessment & On-site written exam, 4 hours under invigilation \\ 

        Type of assessment details & Individual written examination of theoretical exercises typically divided into 3-5 exercises with subquestions, with a mixture of open questions and calculations. \\ 

        Exam registration requirements & Approval of one case, prepared and presented in groups. The case consist of a theoretical or practical assignment regarding a specific unit operation, process line or equivalent. \\ 

        Aid & All aids allowed
        As the exam is an ITX-exam, the University will make computers available to students at the ITX-exam.
        
        Students are not permitted to bring digital aids like computers, tablets, calculators, mobile phones etc.
        
        Books, notes, and similar materials can be brought in paper form or uploaded before the exam and accessed digitally from the ITX computer. Read more about this at Study Information \\ 

        Marking scale & 7-point grading scale \\ 

        Censorship form & External censorship \\ 

        Re-exam & Same as ordinary exam.

        If requirements for examination are not met as a result of non-approved case, the revised case must be handed in at individual basis at the latest 3 weeks prior to the reexamination, and the case must be approved at the latest 1 week prior to the reexamination.
        
        If 10 or fewer register for the reexamination the examination form will be oral. The oral exam will be 30 minutes, no preparation time and all aids allowed.
        
         \\ 
    \end{tabular}
\end{table}

\textbf{Criteria for exam assessment}
See Learning Outcome.
