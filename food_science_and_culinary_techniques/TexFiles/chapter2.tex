\chapter{Basic Knowledge Recap}
\setlength{\headheight}{12.71342pt}
\addtolength{\topmargin}{-0.71342pt}

This section of the course notes is designed to streamline access to the key findings from each reading material (RM), providing a concise and accessible overview of essential information. Created through experimentation with various AI platforms, this chapter also serves to enhance my prompt engineering skills, exploring diverse methods of note-taking for maximum efficiency and clarity. The procedures for creating these summaries have varied, but all methods share a common approach: each RM has been fully read, with summaries and notes prepared after completing each respective subsection. By using these AI-co-op'ed approaches, these notes aim to be both a reliable reference and a resource for continuous improvement in capturing complex microbiology concepts.

\section{1\texorpdfstring{\textsuperscript{st}}{st} RM}
\subsection*{Introduction}
Chemistry evolved over a millennia, merging with biology to form diverse fields like \textbf{food chemistry}, which focuses on sustenance and nutrition. This chapter outlines chemistry's relation to food sciences and its educational context, addressing challenges like increased teaching loads and exploring new opportunities \cite*{BKR_01}.

\subsubsection*{On the Need for Chemistry}
\textbf{Food chemistry} studies the composition, structure, and properties of food and their transformations. It builds on general chemistry, including \textbf{organic}, \textbf{physical}, and \textbf{inorganic} branches. This chapter surveys these areas and introduces key terms for future topics \cite*{BKR_01}.

\subsection*{Organic Chemistry}
\textbf{Organic chemistry} focuses on carbon compounds and \textbf{covalent bonds} with hydrogen. Hydrocarbon nomenclature (e.g., \textbf{alkanes}, \textbf{alkenes}, and \textbf{alkynes}) introduces systematic naming. Topics include bonding theories, \textbf{valence}, and molecular conformation constraints \cite*{BKR_01}.

\subsubsection*{Functional Group Chemistry}
Organic compounds belong to families like \textbf{alkanes}, \textbf{alcohols} (R-OH), and \textbf{carboxylic acids} (R-COOH). Functional groups have consistent behaviour, e.g., \textbf{hydroxyl} groups act similarly across compounds. Their reactivity may vary with the R group, as in alcohol vs. phenol \cite*{BKR_01}.

\subsubsubsection*{Aromatic Compounds}
\textbf{Aromaticity} in compounds like benzene involves \textbf{electron delocalisation} and resonance. Conjugated double bonds in unsaturated lipids and fat-soluble vitamins contribute to their color and oxidative sensitivity. Free radicals drive polymerization of natural monomers like \textbf{isoprene} \cite*{BKR_01}.

\subsubsubsection*{Organic Reaction Mechanisms}
Organic reactions include \textbf{addition}, \textbf{substitution}, and \textbf{rearrangement}, often occurring in raw and processed foods. \textbf{Free radical} reactions drive polymerization and impact \textbf{lipid oxidation}, photochemical processes, and ageing-related deterioration \cite*{BKR_01}.

\subsubsubsection*{Stereochemistry}
\textbf{Stereochemistry} studies chiral compounds, often with asymmetric carbons attached to four groups. Mole- cules with \textbf{N} stereogenic centers yield ($2^N$) stereoisomers. Chirality includes helices lacking stereogenic centers. Chiral molecules rotate plane-polarized light, but \textbf{racemic mixtures} show no net optical activity \cite*{BKR_01}.

\subsection*{Physical Chemistry}
\textbf{Physical chemistry} explores material transformations, focusing on three key themes: \textbf{thermodynamics}, \textbf{chemical kinetics}, and \textbf{quantum mechanics} \cite*{BKR_01}.

\subsubsection*{Thermodynamics}
\textbf{Thermodynamics} examines energy forms, transformations, and efficiency. The \textbf{first law} states energy is conserved, expressed as ($U=Q+W$). The \textbf{second law} highlights entropy increase, with \textbf{Gibbs free energy} ($\Delta G $) predicting spontaneity. Applications include \textbf{reaction calorimetry} and energy content in food \cite*{BKR_01}.

\subsubsection*{Chemical Kinetics}
\textbf{Chemical kinetics} studies reaction rates, influenced by factors like concentration, temperature, and catalysts. Reaction rates follow Equation \ref*{eq:CK_rate}. 
\begin{equation} 
    \text{rate}=k[A]^x[B]^y
    \label{eq:CK_rate}
\end{equation}

With ($k$) described by the \textbf{Arrhenius equation}  which can be seen from Equation \ref*{eq:Arrhenius} 
\begin{equation}
    \ln k=\ln k_0- \frac{\Delta E^\#}{RT} 
    \label{eq:Arrhenius}    
\end{equation}

The \textbf{Eyring theory} relates $k$ to $\Delta G^\#$, $\Delta H^\#$, and $\Delta S^\#$, emphasizing transition states \cite*{BKR_01}.

\subsection*{Inorganic Chemistry}
\textbf{Inorganic chemistry} covers non-carbon elements, including metals, nonmetals, and metalloids. Periodic trends, such as \textbf{atomic radii}, \textbf{ionization energies}, and \textbf{electronegativity}, arise from effective nuclear charge and electron shielding \cite*{BKR_01}.

\subsubsection*{Chemical Bonding}
\textbf{Chemical bonding} can be described with \textbf{Lewis structures} which illustrates electron sharing or transfer. The \textbf{octet rule} explains covalent bonds, while differences in electronegativity lead to polarized or ionic bonds. \textbf{Resonance} applies to compounds like nitrate, where multiple structures describe electron distribution \cite*{BKR_01}.

\subsubsection*{The Shapes of Molecules}
Molecular shapes are predicted using the \textbf{VSEPR model}, based on repulsions between electron pairs. Geometries like \textbf{linear}, \textbf{trigonal planar}, \textbf{tetrahedral}, and \textbf{octahedral} depend on bonded atoms and lone pairs, as seen in molecules like \textbf{$H_2O$} and \textbf{$NH_3$} \cite*{BKR_01}.

\subsubsection*{Valence}
\textbf{Valence-bond theory} explains covalent bonds through orbital overlap. Some valencies require \textbf{hybrid orbitals} (e.g., \textbf{$sp^1$}, \textbf{$sp^2$}, \textbf{$sp^3$}), formed via electron promotion and orbital mixing. Orbital shapes are influenced by electron repulsion \cite*{BKR_01}.

\subsubsection*{Molecular Orbital Theory}
\textbf{Molecular orbital theory} explains bonding via atomic orbital combinations forming equal numbers of bonding $\sigma$ and anti-bonding $\sigma^*$ orbitals. Molecular orbitals follow \textbf{Pauli's exclusion principle} and fill singly before pairing. Diatomic oxygen illustrates these principles \cite*{BKR_01}.

\subsection*{Food Chemistry}
\subsubsection*{Definition and Scope}
\textbf{Food chemistry} applies chemistry principles to food systems, studying macroconstituents (e.g., \textbf{water}, \textbf{carbohydrates}) and microconstituents (e.g., \textbf{vitamins}, \textbf{additives}). Post-WWII advances improved \textbf{shelf-life}, \textbf{packaging}, and analysis of toxicants. Food sciences ensure nutritious, safe, and affordable food \cite*{BKR_01}.

\subsubsection*{Areas of Expertise Required by the Institute of Food Technology}
The \textbf{Institute of Food Technology} identifies five core competencies for food scientists: \textbf{chemistry}, \textbf{analysis}, \textbf{nutrition}, \textbf{microbiology}, and \textbf{engineering} \cite*{BKR_01}.

\subsubsection*{Chemistry and the Food System}
\textbf{Food chemistry} encompasses all levels of the food system, from \textbf{soil pH} in agronomy to the effects of \textbf{pasteurization}. Its scope includes \textbf{harvesting}, \textbf{processing}, \textbf{packaging}, and \textbf{distribution}, as well as studying \textbf{ingredient behaviour} during manufacture \cite*{BKR_01}.

\section{2\texorpdfstring{\textsuperscript{nd}}{nd} RM}
\subsubsection*{Osmosis and Osmotic Pressure}
\textbf{Osmosis} involves water moving to solute-rich areas across membranes. \textbf{Osmotic pressure} drives this process, dehydrating microbes in salted foods. Preservation methods like salting meats and sugaring jams rely on this principle, as seen in \textbf{beef jerky} and jellies \cite*{BKR_02}.

\subsection*{Carbohydrates}
\subsubsection*{Foods High in Carbohydrates}
\textbf{Carbohydrates}, including sugars, starches, and fibers, primarily originate from plants. Sources include \textbf{grains}, \textbf{legumes}, \textbf{fruits}, and \textbf{vegetables}. Exceptions are \textbf{milk}, containing lactose, and animal muscles, storing glycogen. Table sugar comes from \textbf{sugar cane} and \textbf{sugar beets}, while honey is floral nectar \cite*{BKR_02}.

\subsubsection*{Composition of Carbohydrates}
\textbf{Carbohydrates} consist of \textbf{carbon}, \textbf{hydrogen}, and \textbf{oxygen}, following the formula $C_n(H_2O)_n$. Synthesized via \textbf{photosynthesis}, they form saccharides classified as \textbf{monosaccharides}, \textbf{disaccharides}, \textbf{oligosaccharides}, and \textbf{polysaccharides} based on saccharide units \cite*{BKR_02}.

\subsubsection*{Monosaccharides}
\textbf{Monosaccharides} are simple sugars; common types include \textbf{pentoses} (ribose, arabinose) and \textbf{hexoses} (glucose, fructose, galactose). The ending \textit{-ose} indicates that the compound is a sugar \cite*{BKR_02}.

\subsubsubsection*{Ribose and Arabinose}
\textbf{Ribose} is vital in \textbf{DNA}, \textbf{RNA}, and \textbf{ATP}. It also contributes to \textbf{vitamin B\textsubscript{2}}. \textbf{Arabinose} supports the structure of vegetable \textbf{gums} and \textbf{fibers} \cite*{BKR_02}.

\subsubsubsection*{Glucose}
\textbf{Glucose} is the most common \textbf{hexose} in foods and blood. It is found in fruits, honey, and corn syrup. Refined glucose (\textbf{dextrose}) is used in \textbf{candies}, \textbf{baked goods}, and \textbf{ alcoholic beverages}. It is the main component of corn syrup, where it is made by hydrolyzing cornstarch \cite*{BKR_02}.

\subsubsubsection*{Fructose}
\textbf{Fructose}, or \textbf{fruit sugar}, is the sweetest sugar, found in fruits and honey. It causes unwanted properties as \textbf{stickiness}, \textbf{over-browning}, and lowers \textbf{freezing points}. \textbf{High-fructose corn syrup} is widely used in soft drinks \cite*{BKR_02}.

\subsubsubsection*{Galactose}
Is not often found in its free form, but is a component of \textbf{lactose}. A derivative, \textbf{galacturonic acid} is very important in fruits ripening process  \cite*{BKR_02}.

\subsubsection*{Disaccharides}
A disaccharide is formed by two monosaccharides linked together. In this next section, sucrose, lactose and maltose will briefly by discussed \cite*{BKR_02}.

\subsubsubsection*{Sucrose}
Most commonly known as \textbf{table sugar}, and is in its chemical form a disaccharide composed of \textbf{glucose} and \textbf{fructose} \cite*{BKR_02}.

\subsubsubsection*{Lactose}
Most commonly known as \textbf{milk sugar}, and is in its chemical form a disaccharide composed of \textbf{glucose} and \textbf{galactose}. Many people is unable to produce the enzyme \textbf{lactase}, which responsible for \textbf{breaking down} lactose, thus making them \textbf{lactose intolerant}. The characteristic symptoms is \textbf{bloating} and \textbf{abdominal pain}. In some \textbf{fermented milk products}, bacteria break down the lactose to \textbf{lactic acid}, resulting in a product some lactose intolerant individuals can tolerate \cite*{BKR_02}.

\subsubsubsection*{Maltose}
Also called \textbf{malt sugar}, is a disaccharide composed of \textbf{two glucose} units. It is primarily used in the production of \textbf{beer} and \textbf{breakfast cereals} \cite*{BKR_02}.

\subsubsection*{Oligosaccharides}
\textbf{Oligo} comes from Greek and means \textbf{few}. Oligosaccharides are composed of 3-10 monosaccharides. The two most common oligosaccharides are \textbf{raffinose} and \textbf{stachyose}. These are found in \textbf{legumes} and \textbf{cruciferous vegetables}. They are not digested in the small intestine, but are fermented by the gut microbiota in the large intestine \cite*{BKR_02}. Fructo-oligosaccharides (FOS) are also a type of oligosaccharides, and are found in e.g. \textbf{onions} and \textbf{garlic}. They are used as \textbf{pre-biotics} \cite*{BKR_02}.




