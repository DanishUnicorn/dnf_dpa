\chapter{Lecture Exercises}
\setlength{\headheight}{22.94003pt}
\addtolength{\topmargin}{-10.22661pt}




\section{Notes from slides}

% 01_slide
- Bacteria can survive in milk if it is not pasteurized
    - Not nesecarily harmful, but can cause spoilage e.g. quality of milk due to proteolysis

- Pasteurization not only kills/inactivates bacteria, but also enzymes

- Pasteurization extends shelf life of milk

- Some psychrotrophic bacteria can survive pasteurization
    - Gram-negative bacteria can produce heat-stable proteases

- Heat inactivation of extracellular proteinases from \textit{Pseudomonas fluorescens}, \textit{Pseudomonas} spp., and \textit{Achromobacter} spp. is not complete unless heat treated at 130\textdegree C, D-values [s] of 630, 160, and 510, respectively, and with a Q\subscript{10} of 2.1, 1.9, and 2.1, respectively \cite*{s01_protein_degradation_in_bovine_milk}

% 02_slide
- Denaturation: The process of partial or total alteration of the native structure of a macromolecule resulting from a loss in tertiary or tertiary and secondary structure that is a consequence of the disruption of stabilizing weak bonds

- Major whey proteins
    - $\beta$-lactoglobulin
    - $\alpha$-lactalbumin


% 04_slide
- The protein content-, and composition of milk depends on various factors
    - Most important:
        - Breed
        - Lactation stage
        - Genetic variants
        - Cell count    
    - Of significance, but less important:
        - Parity
        - Season
        - Milk yield
        - Feeding
- Cell count and microbiology is tested at arrival of the milk compared
- There are two major groups of milk proteins
    - Caseins       - apx. 80\% of total protein
        - $\alpha\subscript{s1}$-casein
        - $\alpha\subscript{s2}$-casein
        - $\beta$-casein
        - $\kappa$-casein
    - Whey proteins - apx. 20\% of total protein
        - $\alpha$-lactalbumin
        - $\beta$-lactoglobulin
        - Bovine serum albumin
        - Immunoglobulins
        - Enzymes

% 05_slide
- Mastitis has a negative effect on the protein composition of milk, appx. 10\% of the protein is lost.
    - Total casein content is reduced by 10\%
        - $\alpha\subscript{s1}$-casein is reduced by 26-75\%
        - $\beta$-casein is reduced by 26-75\%
        - $\kappa$-casein is increased by 101-1000x
    
    - Total whey protein content is increased by 101-1000x
        - $\beta$-lactoglobulin is reduced by 26-75\%
        - $\alpha$-lactalbumin is reduced by 26-75\%
        - Immunoglobulins are increased by 101-1000x
        - Proteose-peptones are increased by 11-100x
        - Serum albumin is increased by 101-1000x
        - Lactoferrin is increased by 101-1000x

% 07_slide
- Amino acid composition analysis of milk proteins can be done by HPLC-FLD

- Protein separation and purification can be done by column chromatography

- Milk proteins can be analysed by RP-HPLC











