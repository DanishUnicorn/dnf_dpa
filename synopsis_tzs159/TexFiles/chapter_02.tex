\setcounter{chapter}{2}
\setcounter{section}{0}
\section*{Synposis for Question 09}

\section{Introduction}
Milk is an important nutritional source for humans, and serves as the basis for a variety of dairy products. As the main ingredient in dairy products, the microbiological and compositional quality of  the raw milk is of great importance and is directly correlated with the quality of the final dairy product \cite*{a01_protein_degradation_in_bovine_milk}. 

Proteins play an essential role in determining texture, flavour, and functional propperties. If the proteins present in milk are degraded, either enzymatically or due to microbial activity, product quality can suffer significantly \cite*{a02_proteases_and_protein_degradation}. Upholding a high standard for milk is therefore not only of importance with respect to consumer acceptance, but also of economic relevance \cite*{a01_protein_degradation_in_bovine_milk}.




%Milk and dairy products are nutritionally important in the diet worldwide. The microbiological quality of raw milk is essential for the quality of the final dairy product. Quality assurance in milk production at herd level is therefore of great economic importance for both the dairy producer and the dairy industry \cite*{a01_protein_degradation_in_bovine_milk}.

%Degradation of proteins by proteinases in the milk negatively affects cheese yield, but endogeneous proteinases have been found to contribute to the cheese ripening process through casein hydrolysis \cite*{a02_proteases_and_protein_degradation}

%Mastitis is a major problem for dairy producers, as it involves great financial losses. The effect of udder health on the yield and quality of milk and, consequently, on cheese production and quality has been established \cite*{a03_proteolysis_in_milk_during_mastitis}.


\section{Milk Composition}
Milk is a complex liquid whose composition includes a variety of dilute salts, the simple sugar, mainly lactose, and vitamins where fat is emulsified as globules \cite*{a04_bovine_milk_in_human_nutrition}. Proteins in milk is mostly present in the form of casein micelles, which are colloidal aggregates of thousands of molecules \cite*{b01_milk_biochemistry}.

\vspace{0.5em}
Though the protein content (approximately 3.2\% \cite*{s09_milk_secretion_and_ejection}) of milk is relatively low, compared to that of fat globules \cite*{b01_milk_biochemistry}, this synopsis will focus on proteins and their degradation in milk.

\vspace{0.5em}
The milk protein content and composition is influenced by various factors, such as breed, lactation stage, genetic variants, and cell count \cite*{s04_protein_fraction_in_milk}. Futhermore, the protein content of milk is divided into two major groups: caseins and whey proteins. Caseins make up approximately 80\% of the total protein content, while whey proteins make up the remaining 20\% \cite*{s04_protein_fraction_in_milk}. The composition of milk casein proteins consists of $\alpha_{s1}$-casein, $\alpha_{s2}$-casein, $\beta$-casein, and $\kappa$-casein, in respective order \cite*{s04_protein_fraction_in_milk}. The casein proteins are present in structures of micelles and are relatively hydrophobic fibrous proteins \cite*{b01_milk_biochemistry}. Of the existing whey proteins in milk, the major constituents are $\alpha$-lactalbumin, $\beta$-lactoglobulin, bovine serum albumin, immunoglobulins, and enzymes, in respective order \cite*{s04_protein_fraction_in_milk}.




%- There are two major groups of milk proteins \cite*{s04_protein_fraction_in_milk}.
%\begin{itemize}
 %   \item Caseins       - apx. 80\% of total protein
  %      \subitem $\alpha_{s1}$-casein
   %     \subitem $\alpha_s2$-casein
    %    \subitem $\beta$-casein
     %   \subitem $\kappa$-casein
%
 %   \item Whey proteins - apx. 20\% of total protein
  %      \subitem $\alpha$-lactalbumin
   %     \subitem $\beta$-lactoglobulin
    %    \subitem Bovine serum albumin
     %   \subitem Immunoglobulins
      %  \subitem Enzymes
%\end{itemize}


\section{Protein Degradation}
While the somatic cell count (SCC) in milk is an important indicator of udder health, it is used to monitor the presence of mastitis in dairy cows \cite*{s05_mastitis_complex}. Mastitis is an inflammation of the mammary gland, and is a major problem for dairy producers, as an increase in SCC corresponds to an increase in the proteolytic potential in milk \cite*{a02_proteases_and_protein_degradation}. This protein degradation in milk can have negative effects on yield and quality of dairy products, such as cheese \cite*{a02_proteases_and_protein_degradation}.


\subsection{Reasons for Protein Degradation}
There are various reasons for the degradation of proteins in milk, e.g. microbial proteases, endogenous proteases, and heat treatment \cite*{a02_proteases_and_protein_degradation}. 

\paragraph*{Microbial Proteases}
    Some psychrotrophic bacteria can survive pasteurization, as they can produce heat-stable proteases \cite*{s01_heat_treatment_1}. These proteases will be active, even after heat treatment, and can initiate the degradation of the casein- and whey proteins in milk by proteolysis \cite*{b02_heat_induced_changes_in_milk}. This is predominantly the case for psychrotrophic gram-negative bacteria \cite*{s01_heat_treatment_1}.
    If \textit{Pseudomonas fluorescens} is present, normal pasteurization will be insufficient and the bacterias extracellular proteinases will degrade the proteins by proteolysis \cite*{s01_heat_treatment_1}.

\paragraph*{Endogenous Proteases}
    There are many enzymes with milk at its natural habittat, such as plasmin, cathepsin D, and cathepsin B, which can degrade the proteins in milk \cite*{a02_proteases_and_protein_degradation}. Plasmin is the most important and predominant protease in milk, and is largelt responsible for the degradation of casein proteins. The quantity of plasmins is correlated with mastitis, analysis of the raw milk is therefore imperative \cite*{a02_proteases_and_protein_degradation}.

\paragraph*{Heat Treatment}
    Though the primary objective of heat treatments as pasteurization and ultra high temperature pasteurization (UHT-pasteurization) is to kill pathogenic and spoilage bacteria, the temperature does not differentiate by microorganisms, it kills/inactivates all microorganisms which is not sufficiently heat resistant \cite*{b02_heat_induced_changes_in_milk}.

\subsection{Susceptible Proteins}
\paragraph*{Microbial Proteases}
    The proteins most susceptible to proteolysis are the caseins, as they have a high content of proline, which makes them susceptible to proteolysis \cite*{b03_milk_proteins}.

\paragraph*{Endogenous Proteases}
    

\paragraph*{Heat treatment}
Caseins have a high content of proline, for $\alpha_{s1}$-, $\alpha_{s2}$-, $\beta$-, and $\kappa$-casein, the proline content is 17, 10, 35, and 20 residues per mole, respectively \cite*{b03_milk_proteins}. The high proline content in the caseins results in a low content of $\alpha$-helix or $\beta$-sheet structures which makes them susceptible to proteolysis \cite*{b03_milk_proteins}.



\section{Consequences of Proteolysis in Milk Products}
%Say briefly what the consequences are for the different dairy products, i.e. cheese, yogurt, butter, ice cream, whey, etc.


\section{Methods for Analysis}

\subsection{Determining Protein Degradation}

\subsection{Preventing Protein Degradation}

\subsubsection{Dairy Plant Level}

\subsubsection{Farm Level}



\section{Conclusion}

