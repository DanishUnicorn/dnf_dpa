\setcounter{chapter}{2}
\setcounter{section}{0}
\section*{Synposis for Question 09}

\section{Introduction}
Milk is an important nutritional source for humans, and serves as the basis for a variety of dairy products. As the main ingredient in dairy products, the microbiological and compositional quality of  the raw milk is of great importance and is directly correlated with the quality of the final dairy product \cite*{a01_protein_degradation_in_bovine_milk}. 

Proteins play an essential role in determining texture, flavour, and functional propperties. If the proteins present in milk are degraded, either enzymatically or due to microbial activity, product quality can suffer significantly \cite*{a02_proteases_and_protein_degradation}. Upholding a high standard for milk is therefore not only of importance with respect to consumer acceptance, but also of economic relevance \cite*{a01_protein_degradation_in_bovine_milk}.


\section{Milk Composition}
Milk is a complex liquid whose composition includes a variety of dilute salts, the simple sugar, mainly lactose, and vitamins where fat is emulsified as globules \cite*{a04_bovine_milk_in_human_nutrition}. Proteins in milk is mostly present in the form of casein micelles, which are colloidal aggregates of thousands of molecules \cite*{b01_milk_biochemistry}.

\vspace{0.5em}
Though the protein content (approximately 3.2\% \cite*{s09_milk_secretion_and_ejection}) of milk is relatively low, compared to that of fat globules \cite*{b01_milk_biochemistry}, this synopsis will focus on proteins and their degradation in milk.

\vspace{0.5em}
The milk protein content and composition is influenced by various factors, such as breed, lactation stage, genetic variants, and cell count \cite*{s04_protein_fraction_in_milk}. Futhermore, the protein content of milk is divided into two major groups: caseins and whey proteins. Caseins make up approximately 80\% of the total protein content, while whey proteins make up the remaining 20\% \cite*{s04_protein_fraction_in_milk}. The composition of milk casein proteins consists of $\alpha_{s1}$-casein, $\alpha_{s2}$-casein, $\beta$-casein, and $\kappa$-casein, in respective order \cite*{s04_protein_fraction_in_milk}. The casein proteins are present in structures of micelles and are relatively hydrophobic fibrous proteins \cite*{b01_milk_biochemistry}. Of the existing whey proteins in milk, the major constituents are $\alpha$-lactalbumin, $\beta$-lactoglobulin, bovine serum albumin, immunoglobulins, and enzymes, in respective order \cite*{s04_protein_fraction_in_milk}.




%- There are two major groups of milk proteins \cite*{s04_protein_fraction_in_milk}.
%\begin{itemize}
 %   \item Caseins       - apx. 80\% of total protein
  %      \subitem $\alpha_{s1}$-casein
   %     \subitem $\alpha_s2$-casein
    %    \subitem $\beta$-casein
     %   \subitem $\kappa$-casein
%
 %   \item Whey proteins - apx. 20\% of total protein
  %      \subitem $\alpha$-lactalbumin
   %     \subitem $\beta$-lactoglobulin
    %    \subitem Bovine serum albumin
     %   \subitem Immunoglobulins
      %  \subitem Enzymes
%\end{itemize}


\section{Protein Degradation}
While the somatic cell count (SCC) in milk is an important indicator of udder health, it is used to monitor the presence of mastitis in dairy cows \cite*{s05_mastitis_complex}. Mastitis is an inflammation of the mammary gland, and is a major problem for dairy producers, as an increase in SCC corresponds to an increase in the proteolytic potential in milk \cite*{a02_proteases_and_protein_degradation}. This protein degradation in milk can have negative effects on yield and quality of dairy products, such as cheese \cite*{a02_proteases_and_protein_degradation}.


\subsection{Reasons for Protein Degradation}
There are various reasons for the degradation of proteins in milk, e.g. microbial proteases, endogenous proteases, and heat treatment \cite*{a02_proteases_and_protein_degradation}. 

\paragraph*{Microbial Proteases}
    Some psychrotrophic bacteria can survive pasteurization, as they can produce heat-stable proteases \cite*{s01_heat_treatment_1}. These proteases will be active, even after heat treatment, and can initiate the degradation of the casein- and whey proteins in milk by proteolysis \cite*{b02_heat_induced_changes_in_milk}. This is predominantly the case for psychrotrophic gram-negative bacteria \cite*{s01_heat_treatment_1}.
    If \textit{Pseudomonas fluorescens} is present, normal pasteurization will be insufficient and the bacterias extracellular proteinases will degrade the proteins by proteolysis \cite*{s01_heat_treatment_1}.

\paragraph*{Endogenous Proteases}
    There are many enzymes with milk at its natural habittat, such as plasmin, cathepsin D, and cathepsin B, which can degrade the proteins in milk \cite*{a02_proteases_and_protein_degradation}. Plasmin is the most important and predominant protease in milk, and is largelt responsible for the degradation of casein proteins. The quantity of plasmins is correlated with mastitis, analysis of the raw milk is therefore imperative \cite*{a02_proteases_and_protein_degradation}.

\paragraph*{Heat Treatment}
    Though the primary objective of heat treatments as pasteurization and ultra high temperature pasteurization (UHT-pasteurization) is to kill pathogenic and spoilage bacteria, the temperature does not differentiate by microorganisms, it kills/inactivates all microorganisms which is not sufficiently heat resistant \cite*{b02_heat_induced_changes_in_milk}.

\subsection{Susceptible Proteins}
\paragraph*{Microbial Proteases}
    The proteins most susceptible to microbial derived enzymes, resulting in proteolysis are the caseins, as their high proline content makes them more susceptible to proteolysis \cite*{b03_milk_proteins}.

\paragraph*{Endogenous Proteases}
The most susceptible proteins to endogenous proteases are the caseins $\alpha-casein_{s1}$, $\beta$-casein and the whey protein $\alpha$-lactalbumin. The decrease for these three proteins are percentagely the same (26-75\%), but the content of the caseins is approximately 10 times higher, therefore, the overall quantitative loss is greatest for caseins \cite*{s05_mastitis_complex}.


\paragraph*{Heat treatment}
Caseins have a high content of proline, for $\alpha_{s1}$-, $\alpha_{s2}$-, $\beta$-, and $\kappa$-casein, the proline content is 17, 10, 35, and 20 residues per mole, respectively \cite*{b03_milk_proteins}. The high proline content in the caseins results in a low content of $\alpha$-helix or $\beta$-sheet structures makes them structurally more prone to denaturation and aggregation under heat treatment \cite*{b03_milk_proteins}.


\section{Consequences of Proteolysis in Milk Products}
Severely heat treating milk can result in degradation of proteins in milk which in turn can have negative effects on the quality of the final dairy product, e.g. cheese not ripening properly due to high moisture and excessive syneresis in yogurt   \cite*{b02_heat_induced_changes_in_milk}. Another consequence of heat treating milk is maillard reaction which can result in a brownish color and a cooked taste \cite*{a08_shelf_life_of_heat_treated_dairy_products}. Enzymatic activity from plasmin has also been shown to have a direct impact on both bitterness, gellation and shortens shelf-life \cite*{a08_shelf_life_of_heat_treated_dairy_products}. A combination of high enzymatic activity, resulting in more available amino acids, and high temperature treatments can result in a more pronounced maillard reaction \cite*{a08_shelf_life_of_heat_treated_dairy_products}.

If there is a reduced amount of casein proteins in milk, leading to reduced cheese yield, since caseins are the primary proteins responsible for curd formation \cite*{a02_proteases_and_protein_degradation}. The reduction in casein proteins will also affect the cheese ripening process, as the endogeneous proteinases contribute to the cheese ripening process through casein hydrolysis \cite*{a02_proteases_and_protein_degradation}.


\section{Methods for Analysis}

\subsection{Determining Protein Degradation}
SDS-PAGE, a gel electrophoresis method, has been shown to indicate protein degradation in milk \cite*{a02_proteases_and_protein_degradation}. By staining the gel, degraded proteins apear as distinct bands on the plate. Smaller peptides travel further through the gel matrix, allowing differentiation of the protein size, depending on the travel length  \cite*{a02_proteases_and_protein_degradation}.

The fluorescamine method is another technique for analysing protein degradation in milk. It is based on the reaction of fluorescamine with primary amines, which produces a fluorescent product. The emitted fluorescence can be measured spectrometrically, and the intensity is proportional to the quantity of primary amines present in the sample, giving an indication of protein degradation in milk \cite*{a02_proteases_and_protein_degradation}.

Other spectrometrical methods, such as seize-exclusive chromatography HPLC (SEC-HPLC), reverse-phase HPLC (RP-HPLC), and liquid chromatography-mass spectrometry (LC-MS) can also be used to analyse protein degradation in milk \cite*{s07_hplc_milk_components,a09_proteomics}. An optimized version of LC-MS HPLC has been suggested to provide a novel insight in the complexity of the main milk proteins \cite*{a09_proteomics}.

\subsection{Preventing Protein Degradation}
Good raw milk quality is essential for any use. When focusing on preventing protein degradation raw milk has to be collected at farms that uphold strict hygienic standards and ensures immediate storage at low temperatures. This ensures an unfavourable environment for bacteria, particularly psychrotrophic species \cite*{a08_shelf_life_of_heat_treated_dairy_products}. Before the raw milk is being processed at the dairy plant, microbiological testing should be performed to ensure low bacterial counts and low somatic cell levels. After passing microbiological testing, the raw milk should be pasteurised promptly to inactivate native and microbiological enzymes before further processing \cite*{a08_shelf_life_of_heat_treated_dairy_products}. 


\section{Conclusion}

