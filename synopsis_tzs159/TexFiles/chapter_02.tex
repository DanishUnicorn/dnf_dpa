\setcounter{chapter}{2}
\setcounter{section}{0}
\section*{Synposis for Question 09}

\section{Introduction}
Milk is an important nutritional source for humans, and serves as the basis for a variety of dairy products. As the main ingredient in dairy products, the microbiological and compositional quality of  the raw milk is of great importance and is directly correlated with the quality of the final dairy product \cite*{a01_protein_degradation_in_bovine_milk}. 

Proteins play an essential role in determining texture, flavour, and functional propperties. If the proteins present in milk are degraded, either enzymatically or due to microbial activity, product quality can suffer significantly \cite*{a02_proteases_and_protein_degradation}. Upholding a high standard for milk is therefore not only of importance with respect to consumer acceptance, but also of economic relevance \cite*{a01_protein_degradation_in_bovine_milk}.




%Milk and dairy products are nutritionally important in the diet worldwide. The microbiological quality of raw milk is essential for the quality of the final dairy product. Quality assurance in milk production at herd level is therefore of great economic importance for both the dairy producer and the dairy industry \cite*{a01_protein_degradation_in_bovine_milk}.

%Degradation of proteins by proteinases in the milk negatively affects cheese yield, but endogeneous proteinases have been found to contribute to the cheese ripening process through casein hydrolysis \cite*{a02_proteases_and_protein_degradation}

%Mastitis is a major problem for dairy producers, as it involves great financial losses. The effect of udder health on the yield and quality of milk and, consequently, on cheese production and quality has been established \cite*{a03_proteolysis_in_milk_during_mastitis}.


\section{Milk Composition}



%Bovine milk contains the nutrients needed for growth and development of the calf, and is a resource of lipids, proteins, amino acids, vitamins and minerals. It contains immunoglobulins, hormones, growth factors, cytokines, nucleotides, peptides, polyamines, enzymes and other bioactive peptides \cite*{a04_bovine_milk_in_human_nutrition}. 

%Milk, alongside meat, is one of the basic animal materials of importance in food processing. Dairy production around the world, including Poland, is dominated by cow milk. This is due to the higher productivity of cows compared to other dairy species. Moreover, it is the most universal raw milk for processing due to its specific content and proportions of proteins, fat, and mineral compounds \cite*{a05_organic_vs_conventional_milk}

%- The protein content-, and composition of milk depends on various factors \cite*{s04_protein_fraction_in_milk}
%\begin{itemize}
%    \item Most important:
 %        \subitem Breed
  %       \subitem Lactation stage
   %      \subitem Genetic variants
    %     \subitem Cell count    

%    \item Of significance, but less important:
 %           \subitem Parity
  %          \subitem Season
   %         \subitem Milk yield
    %        \subitem Feeding
%\end{itemize}

%- Cell count and microbiology is tested at arrival of the milk compared \cite*{s04_protein_fraction_in_milk}.


%- There are two major groups of milk proteins \cite*{s04_protein_fraction_in_milk}.
%\begin{itemize}
 %   \item Caseins       - apx. 80\% of total protein
  %      \subitem $\alpha_{s1}$-casein
   %     \subitem $\alpha_s2$-casein
    %    \subitem $\beta$-casein
     %   \subitem $\kappa$-casein
%
 %   \item Whey proteins - apx. 20\% of total protein
  %      \subitem $\alpha$-lactalbumin
   %     \subitem $\beta$-lactoglobulin
    %    \subitem Bovine serum albumin
     %   \subitem Immunoglobulins
      %  \subitem Enzymes
%\end{itemize}

%- Mastitis has a negative effect on the protein composition of milk, appx. 10\% of the protein is lost \cite*{s05_mastitis_complex}.
%\begin{itemize}
 %   \item Total casein content is reduced by 10\%
  %      \subitem $\alpha_{s1}$-casein is reduced by 26-75\%
   %     \subitem $\beta$-casein is reduced by 26-75\%
    %    \subitem $\kappa$-casein is increased by 101-1000x
%
 %   \item Total whey protein content is increased by 101-1000x
  %  \subitem $\beta$-lactoglobulin is reduced by 26-75\%
   % \subitem $\alpha$-lactalbumin is reduced by 26-75\%
    %\subitem Immunoglobulins are increased by 101-1000x
    %\subitem Proteose-peptones are increased by 11-100x
    %\subitem Serum albumin is increased by 101-1000x
    %\subitem Lactoferrin is increased by 101-1000x
%\end{itemize}


\section{Protein Degradation}
%Elevated SCC can exhibit a negative correlation with both the yields and percentages of milk protein and fat \cite*{a06_organic_milk_production}.

%Bacteria can survive in milk if it is not pasteurized
 %   - Not nesecarily harmful, but can cause spoilage e.g. quality of milk due to proteolysis \cite*{s01_heat_treatment_1}.

%Some psychrotrophic bacteria can survive pasteurization
 %   - Gram-negative bacteria can produce heat-stable proteases \cite*{s01_heat_treatment_1}.

%- Heat inactivation of extracellular proteinases from \textit{Pseudomonas fluorescens}, \textit{Pseudomonas} spp., and \textit{Achromobacter} spp. is not complete unless heat treated at 130\textdegree C, D-values [s] of 630, 160, and 510, respectively, and with a $Q_{10}$ of 2.1, 1.9, and 2.1, respectively \cite*{s01_heat_treatment_1}

%- Denaturation: The process of partial or total alteration of the native structure of a macromolecule resulting from a loss in tertiary or tertiary and secondary structure that is a consequence of the disruption of stabilizing weak bonds \cite*{s02_heat_treatment_2}.



\subsection{Reasons for Protein Degradation}


\subsection{Susceptible Proteins}


\section{Consequences of Proteolysis in Milk Products}
%Say briefly what the consequences are for the different dairy products, i.e. cheese, yogurt, butter, ice cream, whey, etc.


\section{Methods}

\subsection{Determining Protein Degradation}

\subsection{Preventing Protein Degradation}

\subsubsection{Dairy Plant Level}

\subsubsection{Farm Level}



\section{Conclusion}

